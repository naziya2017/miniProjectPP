\documentclass{article}
\usepackage{graphicx} % Required for inserting images

\title{CalConverter and Base Converter: A Mini Project}
\author{Naziya Bano}
\date{October 2023}

\usepackage[a4paper, margin=1in]{geometry} % Adjust the page geometry

\begin{document}

\maketitle

\section{Introduction}
I have created a web application called CalConverter using HTML, CSS, and JavaScript. This application is designed to handle real-world calculations, including addition, subtraction, multiplication, modulus, and trigonometric calculations. It follows the principle of using trigonometric identities with minimal extra operations to perform various tasks.

\section{CalConverter Features}
CalConverter provides the following features:

\begin{itemize}
    \item Addition, Subtraction, Multiplication, and Modulus calculations.
    \item Trigonometric calculations, utilizing trigonometric identities for efficiency.
\end{itemize}

\section{Base Converter}
The second application in this project is the Base Converter. In digital systems, data is often stored in binary, and addresses are represented in hexadecimal. Therefore, there's a need to convert decimal values into various base representations. The Base Converter can convert a value from one base to another, supporting a minimum base of 2 and a maximum base of 36.

\section{Conclusion}
This mini project combines a powerful calculator, CalConverter, with a versatile Base Converter. These tools are essential for various computer science and digital systems applications.

\end{document}
